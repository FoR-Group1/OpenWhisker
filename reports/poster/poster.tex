\documentclass[final]{beamer}

\usepackage{blindtext}
\usepackage{comment}
\usepackage{graphicx}
\usepackage{capt-of}
\usepackage{wrapfig}
% \usepackage[T1]{fontenc}
\usepackage{lmodern}
\usepackage[orientation=portrait,size=a0,scale=1.2]{beamerposter}
\usetheme{gemini}
\usecolortheme{gemini}
\usepackage{booktabs}
\usepackage{tikz}
\usepackage{pgfplots}
\pgfplotsset{compat=1.18}
\usepackage{anyfontsize}
\usepackage[style=ieee,backend=biber,natbib=true]{biblatex}
\AtBeginBibliography{\small}
\graphicspath{ {./figures/} }

\addbibresource{Whiskers.bib}

\footercontent{* This work was partly supported by the EPSRC grant [EP/S023917/1]. \hfill
  \href{mailto:liyou.zhou@outlook.com}{liyou.zhou@outlook.com}}

\newlength{\sepwidth}
\newlength{\colwidth}
\setlength{\sepwidth}{0.025\paperwidth}
\setlength{\colwidth}{0.46\paperwidth}

\title{3D Printer Based Open Source Calibration Platform for Whisker Sensors}
\author{
    Liyou Zhou \and
    Omar Ali \and
    Soumo Emmanuel Arnaud \and\linebreak
    Eden Attenborough \and
    Jacob Swindell \and
    George Davies \and
    Charles Fox
}
\date{\today}
\institute[shortinst]{Department of Computer Science, University of Lincoln}

\newcommand{\separatorcolumn}{\begin{column}{\sepwidth}\end{column}}

\begin{document}

\begin{frame}[t]
\begin{columns}[t]

\separatorcolumn

\begin{column}{\colwidth}
  \begin{block}{Introduction}
    \begin{itemize}
      \item Whisker sensors are a type of tactile sensor increasingly used in robotics. \cite{fotouhiDetectionBarelyVisible2021,struckmeierViTaSLAMBioinspiredVisuoTactile2019,foxTactileSLAMBiomimetic2012}
      \item Calibration of sensors remains a cost prohibitive and time-consuming process. \cite{sullivanTactileDiscriminationUsing2012,fotouhiDetectionBarelyVisible2021,evansWhiskerobjectContactSpeed2010}
      \item We present a low-cost open-source calibration and testing platform for whisker sensors based on an off-the-shelf 3D printer
    \end{itemize}
  \end{block}

  \begin{block}{Hardware Components}
    \begin{enumerate}
        \item \textbf{3D Printer} A Prusa i3 (MK2)~\cite{OriginalPrusaI3}.
        \item \textbf{Whisker Sensor Mount} secures the whisker onto the printer bed.
        \item \textbf{Whisker Sensor} is of a similar design to \cite{stevenson2024whisker}.
        \item \textbf{End Effector} a metal ruler.
    \end{enumerate}
  
    \begin{figure}
      \centering
      \includegraphics[width=0.95\linewidth]{overview}
      \caption{Calibration Platform Overview}
      \label{fig:overview}
    \end{figure}
  \end{block}

  \begin{block}{Software Components}
    \begin{enumerate}
      \item \textbf{Printer Driver} Interface with 3D printer via GCode~\cite{kramerNISTRS274NGCInterpreter2000}.
      \item \textbf{Sensor Driver} Read whisker sensor via serial at 800 Hz.
      \item \textbf{ROS integration} Publishes sensor data to ROS topics and controls the printer via a service interface. Uses foxglove for visualization and rosbag for data acquisition.
    \end{enumerate}

    \begin{figure}
      \centering
      \includegraphics[width=.7\textwidth]{foxglove_visualisation.png}
      \caption{Live visualization of the sensor system via Foxglove}
      \label{fig:foxglove}
    \end{figure}
  \end{block}

  \begin{block}{Open Source}
    All project artefacts are available at \url{https://github.com/FoR-Group1/OpenWhisker}. Would you like to know more? Do you have something we can collaborate on? Scan the QR code below.
    \begin{figure}
      \centering
      \includegraphics[width=.3\columnwidth]{qr_code.png}
      \label{fig:github}
    \end{figure}
  \end{block}

\end{column}

\separatorcolumn

\begin{column}{\colwidth}

  \begin{block}{Calibration Results}
    Printer head is driven to make contact with the whisker sensor at a series of radial distances. Sensor data is recorded along with 3D location of the printer head.

    \begin{figure}
      \centering
      \includegraphics[width=0.49\textwidth]{raw_magnetometer.png}
      \hspace{1pt}
      \includegraphics[width=0.49\textwidth]{3d_printer_raw.png}
      \caption{Raw Magnetometer (left) and 3D Printer Data (right) from 3 consecutive Calibration Routines}
      \label{fig:calibration_routine}
    \end{figure}

    Taking the second derivative of the sensor data, a correlation can be regressed using a polynomial model.
  
    \begin{figure}
      \centering
      \includegraphics[width=.7\columnwidth]{polynomial_regression.png}
      \caption{Polynomial Regression of the relationship between 120th sample of \(\dot{y}\) in each episode and the radial contact distance \(x\)}
      \label{fig:polynomial_regression.png}
    \end{figure}

    If a contact is made within $70 mm$ from the tip of the whisker, the contact location can be accurately inferred to within $2 mm$ by the model.
  \end{block}

  \begin{block}{References}
    \vspace{-.4cm}
    \printbibliography[heading=none]
  \end{block}

\end{column}

\separatorcolumn

\end{columns}
\end{frame}

\end{document}